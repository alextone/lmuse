%% (c) 2012 florian jung
%% we should consider putting this under a proper license. GPL, or
%% some GPL-like documentation license??

%% rules for editing documentation: (READ THIS FIRST)
%% ~~~~~~~~~~~~~~~~~~~~~~~~~~~~~~~~~~~~~~~~~~~~~~~~~~
%%
%% please try to let newly written lines be shorter than 72 characters.
%% minor exceptions are okay, but please not more than 80 chars.
%% comments shall start after character #80 of the line (that is,
%% they shall be "on the right margin")
%%
%% DON'T MIX up changes and reformatting in one commit. when changing
%% stuff, please don't touch the lines before and after your change
%% (that is, do not re-wrap them), even if it will look a bit patchy.
%% this is for being able to easily use diff.
%% when you want to reformat this file, then do it. but don't change
%% anything, as this would be hard to find in a diff. and clearly
%% state in the commit log that you "only" rearranged things.
%%
%% please adhere to the "User's manual" / "Internals" / "Design"
%% partitioning (genereally, don't change the chapters until there
%% is a really good reason for doing so (adding a chapter like
%% "feature requests" as flo did in r1497 IS one).
%% Below that, feel free to change the logical arrangement
%% (making paragraphs to subsections and similar) if you deem it
%% neccessary.
%%
%% Whenever referring to code symbols, constants or source/header
%% files, please use \texttt{someClass::someSymbol} or \texttt{file.cpp}
%% Only specify file paths or namespaces where it would be ambiguous
%% otherwise. Specify 'someClass::' until it would disturb the reader
%% and it is obvious. (watch out to prefix _ with \)
%%
%% Where possible, reference other parts of this documents with
%% \label and \ref. Avoid duplicate information under "Internals" by
%% referring to the appropriate section in "User's manual".
%%
%% Please do no time-stamping of sections. if you need time-stamps,
%% use "svn blame documentation.tex"
%%
%% If you contribute something, feel free to add yourself to \author.
%%
%% If you don't speak LaTeX fluently, a few tips:
%% * \section, \subsection, \subsubsection, \paragraph, \subparagraph
%%   let you create sections etc. just copy-and-paste if unsure.
%% * you must prefix special characters like the underscore with \
%%   (backslash)
%% * \emph{some text} emphasizes the text, printing it italic.
%% * \texttt{some text} displays the text in a typewriter font
%% * \label{someName} creates a label at this position. this doesn't
%%   show up in the pdf. with \ref{someName}, you can reference to this
%%   label. (LaTeX will insert the section number instead of \ref)
%%   For this to work, you might need to recompile the .tex twice.
%%
%%%%%%%%%%%%%%%%%%%%%%%%%%%%%%%%%%%%%%%%%%%%%%%%%%%%%%%%%%%%%%%%%%%%%%%%



\documentclass[a4paper]{report}
\usepackage[T1]{fontenc}
\usepackage[utf8]{inputenc}
\usepackage{lmodern}
\usepackage[english]{babel}
\author{Florian Jung}
\title{MusE Documentation}
\begin{document}

\chapter{User's manual}
someone should really write this!

\section{Basic overview}
\subsection{Tracks}
MusE arranges your music in \emph{tracks} and \emph{parts}. The following
section shall provice you an overview of how things are done with MusE.
If you are or were a Cubase or Cakewalk user, you will feel familiar with
this. There are
MIDI and drum tracks (which are internally MIDI tracks) which can hold
note data, wave tracks which hold wave data, and several kinds of
audio tracks, synthesizers etc.

\paragraph{MIDI tracks}
MIDI and drum tracks hold MIDI event data. They don't differ much,
except that drum tracks offer a special editor which is more suitable
for drum editing.

\paragraph{Wave tracks}
They hold audio data which can be just played back or be piped through
effect plugin chains. They offer automation for these plugins.

\paragraph{Audio inputs}
These provide the path for your audio data from outside into your
project. Set up the physical audio inputs you want to connect your audio
in track with, and then route them to some wave tracks.

                                                                         % TODO: what are audio outs and auxs?
                                                                         %       someone please write this!
                                                                         
\subsection{Parts}
Tracks are split in parts. These contain MIDI events or wave chunks.
You can copy, clone them, move them around, delete them etc.

\subsection{MIDI ports}
Ports provide an abstraction layer for your MIDI synthesizers (which
can be both software and hardware synthesizers). Port are numbered.
In order to produce sound, each MIDI track is assigned to exactly one
MIDI port, to which the MIDI events are then sent.

In the configuration menu, you must map the port numbers to the actual
synth devices (by selecting ALSA or jack midi ports, or synth plugins).

Try left-clicking on the "Ports" column of some MIDI track. If you
use a soft synth, right-clicking the Ports column of the synth or any
track using the synth lets you launch the synthe's GUI.

\section{Tracks and parts}
\subsection{Tracks}
\paragraph{Creation}
You can create a track by either right-clicking in the arranger's track   % TODO: insert screenshot
list and then adding the desired track, or via the edit menu.

\paragraph{Attributes}                                                    % TODO: this should be a list
Tracks have several attributes.
\subparagraph{Mute} If you click on the \emph{Mute} field (denoted with
a "M" column header), the track gets muted and stops producing sound.
\subparagraph{Solo} The solo button ("S" column header) mutes all other   % TODO: soloing chains etc?
tracks.
\subparagraph{Record} The R column "arms" your track for recording.
When you rec-arm your song and have no tracks rec-armed, you won't be
able to record anything. See also the config option "move rec-arm with    % TODO: reference to rec-arm config option
selection".
\subparagraph{Track name} Double-click and you will see.
\subparagraph{Port} For MIDI tracks, this lets you select the MIDI
port to which the events should be routed. This can be your physical
synthesizer or a software synthesizer. For soft synthes, this is the
port the synth is associated to. For other track types, this is disabled.
\subparagraph{Channel} For MIDI tracks, this is the MIDI channel the
output is sent to. For any kind of audio tracks, this is the number of
channels (mono, stereo).
                                                                          % TODO: what's that "T" column?!
\subparagraph{Automation} For audio tracks, this lets you set up the
automation display in the arranger. Clicking this will provide you with
a popup menu with lots of submenus. The submenus let you select the
color you want to associate with the automation parameter. \emph{Clicking % TODO: this is nonstandard
on a submenu} will select or unselect it, making the automation parameter % TODO: put reference to automation section
shown or hidden.

\subparagraph{Clef} For MIDI tracks, you can specify a clef here. This
only affects the score editor.

\paragraph{The trackinfo side bar}
In the arranger and the part editors, you'll have a trackinfo sidebar
on the left side. You can set up track-type specific things there.

\subsubsection{MIDI tracks}
MIDI parts have no automation settings. However, they support various     % TODO: this will hopefully become wrong soon
controllers, program changes etc.. The MIDI trackinfo sidebar lets you
change program, volume, pan and more. Just editing the value in the
boxes will send an event, but not write it to the track. In order to
write it, you will need to click on the corresponding button ("Vol" for
writing down volume information).

\subsubsection{Old style drum tracks}
These are MIDI tracks as well, but with a few differences. They allow
you to map certain drum sounds with different input notes, and you
can change the output settings of a certain "drum instrument" without
having to alter each single event.

However, they have certain limitations: They only can handle 128 sounds
(even if you have more synthes), they aren't really compatible with
MIDI tracks (you can interchange parts between them, but if you touched
the drum list, you'll get unexpected results), you can't set a program
for the used channel and more.

\subsubsection{New style drum tracks}
That's why there will be new-style drum tracks in the next development version. % TODO: in trunk, change this
They are handled exactly like plain MIDI tracks (staying compatible with
them), and offer all of the functionality, though in a different way.
They allow you to re-order the drum map efficiently, you can open parts
from multiple drum tracks in \emph{one} drum editor (MusE will separate
the sounds from different tracks according to your settings, see the
"Window Config" menu), and you can set programs as with normal MIDI tracks.

\subsubsection{Audio tracks}
\paragraph{Effect rack}
On the top of the sidebar, there is an effect rack which allows you to
apply various plugins on the audio. For more information on this, refer
to \ref{plugins}.

\paragraph{Controls}
Lorem ipsum % TODO FINDMICH HIER WEITER


\subsection{Parts}
Within MIDI, drum and wave tracks, you can create \emph{parts}. Parts
are chunks of coherent notes or wave data which can be moved around,
copied, cloned and deleted independent from other parts.

Parts are created by selecting the pencil tool and then drawing onto
the right part area in the arranger. You can move them with the arrow
tool, delete them using the \texttt{DEL} key, and a right-click opens
a popup menu. This menu allows you even more stuff, such as setting
the part's color, saving the part to disk etc.. You can use
\texttt{CTRL+C} and \texttt{CTRL+V} for copying and pasting parts.
\texttt{CTRL+B} pastes the part as a clone. Pressing \texttt{SHIFT}
additionally provides you a dialog which allows you to paste the part
multiple times and set more stuff.

You can also copy parts with the mouse by moving the part with the mouse
while holding down the \texttt{CTRL} key.



\section{Plugins and automation} \label{plugins}
There are several kinds of plugins. First, there are audio plugins, which
can be applied to any track handling audio (that is, inputs, outputs,
wave tracks, synth tracks). Plugins can be added by double-clicking
on an entry in the effect rack in the track info pane (which is shown
at the left side of the arranger when the according track is selected).
Right-clicking them offers a self-explanatory popup menu.

\section{Configuration}
\paragraph{Minimum control period}
Plugins can usually process an arbitrarily small (or large) amount of
samples. If some plugin control value changes continously, to provide
ideal listening experience, MusE would need to call the plugin 44100
times a second, asking for one single value at a time. With the minimum
control period setting, the user can force MusE to ask the plugin for
at least N values. Setting this value to 64 would in this situation
make MusE call the plugin $689 = \frac{44100}{64})$ times a second,
asking for 64 values at a time. While doing this will reduce accuracy
of control changes, it may also reduce CPU usage, because calling
the plugin more often, requesting smaller chunks, is more expensive
than calling it seldomly, requesting larger chunks.
\subparagraph{Recommendation}
If you have no performance problems, or if you want to do the final
downmix of your project, set this to a low value. If you're experiencing
performance problems, increasing this value might help.






\chapter{Internals -- how it works}
This chapter explains how MusE is built internally, and is meant
to be an aid for developers wanting to quickly start up with MusE.
For details on \emph{why} stuff is done please refer to the following
chapter.
\section{User controls and automation}
\subsection{Handling user input}
\paragraph{Plugins and synthesizers}
When the user launches a plugin's GUI, either a MusE-window with
the relevant controls is shown, or the native GUI is launched. MusE
will communicate with this native GUI through OSC (Open Sound Control).

The relevant classes are \texttt{PluginGui}, \texttt{PluginIBase}
(in \texttt{plugin.h}) and \texttt{OscIF} (in \texttt{osc.h}).

If the user changes a slider, first the corresponding control is
disabled, making MusE not steadily update it through automation
while the user operates it. \texttt{PluginIBase::setParam} is called,
which usually writes the control change into the ringbuffer
\texttt{PluginI::\_controlFifo}. (\texttt{PluginI::apply()},
\texttt{DssiSynthIF::getData()} will read this ringbuffer and
do the processing accordingly). Furthermore, the change is written
into a "to be recorded"-list (done by calling \texttt{AudioTrack::recordAutomation}).
This list is processed after recording has finished.                                   %TODO: where and when exactly?

Disabling the controller is both dependent from the current automation
mode and from whether the GUI is native or not.
In \texttt{AUTO\_WRITE} mode, once a slider is touched (for MusE-GUIs) or
once a OSC control change is received (for native GUIs), the control
is disabled until the song is stopped or seeked.

In \texttt{AUTO\_TOUCH} (and currently (r1492) \texttt{AUTO\_READ}, but
that's to be fixed) mode, once a MusE-GUI's slider is pressed down, the
corresponding control is disabled. Once the slider is released, the
control is re-enabled again. Checkboxes remain in "disabled" mode,
however they only affect the recorded automation until the last toggle
of the checkbox. (Example: start the song, toggle the checkbox, toggle
it again, wait 10 seconds, stop the song. This will NOT overwrite the
last 10 seconds of automation data, but everything between the first
and the last toggle.). For native GUIs, this is a bit tricky, because
we don't have direct access to the GUI widgets. That is, we have no
way to find out whether the user doesn't touch a control at all, or
whether he has it held down, but just doesn't operate it. The current
behaviour for native GUIs is to behave like in \texttt{AUTO\_WRITE} mode.

The responsible functions are: \texttt{PluginI::oscControl} and
\texttt{DssiSynthIF::oscControl} for handling native GUIs,
\texttt{PluginI::ctrlPressed} and \texttt{ctrlReleased} for MusE
default GUIs and \texttt{PluginI::guiParamPressed},
\texttt{guiParamReleased}, \texttt{guiSliderPressed} and
\texttt{guiSliderReleased} for MusE GUIs read from a UI file;
\texttt{guiSlider*} obviously handle sliders, while \texttt{guiParam*}
handle everything else which is not a slider. They call
\texttt{PluginI::enableController} to enable/disable it.

Furthermore, on every song stop or seek, \texttt{PluginI::enableAllControllers}
is called, which re-enables all controllers again. The call paths for
this are:
\begin{itemize}
\item For stop: \texttt{Song::stopRolling} calls
                \texttt{Song::processAutomationEvents} calls
                \texttt{Song::clearRecAutomation} calls
                \texttt{Track::clearRecAutomation} calls
                \texttt{PluginI::enableAllControllers}
\item For seek: \texttt{Audio::seek} sends a message ("\texttt{G}") to
                \texttt{Song::seqSignal} which calls
                \texttt{Song::clearRecAutomation} which calls
                \texttt{PluginI::enableAllControllers}                
\end{itemize}




\chapter{Design decisions}
\section{Automation}
As of revision 1490, automation is handled in two ways: User-generated
(live) automation data (generated by the user moving sliders while playing)
is fed into \texttt{PluginI::\_controlFifo}. Automation data is kept
in \texttt{AudioTrack::\_controller}, which is a \texttt{CtrlListList},
that is, a list of \texttt{CtrlList}s, that is, a list of lists of
controller-objects which hold the control points of the automation graph.
The \texttt{CtrlList} also stores whether the list is meant discrete
(a new control point results in a value-jump) or continous (a new control
point results in the value slowly sloping to the new value).

While \texttt{PluginI::\_controlFifo} can be queried very quickly and
thus is processed with a very high resolution (only limited by the
minimum control period setting), the automation value are expensive to
query, and are only processed once in an audio \emph{driver} period.
This might lead to noticeable jumps in value.

This could possibly be solved in two ways:
\paragraph{Maintaining a slave control list}
This approach would maintain a fully redundant slave control list,
similar to \texttt{PluginI::\_controlFifo}. This list must be updated
every time any automation-related thing is changed, and shall contain
every controller change as a tuple of controller number and value.
This could be processed in the same loop as \texttt{PluginI::\_controlFifo},
making it comfortable to implement; furthermore, it allows to cleanly
offer automation-settings at other places in future (such as storing
automation data in parts or similar).

\paragraph{Holding iterators}
We also could hold a list of iterators of the single \texttt{CtrlList}s.
This would also cause low CPU usage, because usually, the iterators only
need to be incremented once. However, it is pretty complex to implement,
because the iterators may become totally wrong (because of a seek in the
song), and we must iterate through a whole list of iterators.

\paragraph{Just use the current data access functions}
By just using the current functions for accessing automation data,
we might get a quick-and-dirty solution, which however wastes way too
much CPU ressources. This is because on \emph{every single frame}, we
need to do a binary search on multiple controller lists.


\chapter{Feature requests}
\section{Per-Part automation and more on automation}                       % by flo
Automation shall be undo-able. Automation shall reside in parts which
are exchangeable, clonable etc (like the MIDI- and Wave-Parts).
Global per-synth/per-audiotrack automation shall also be available, but
this can also be implemented as special case of part automation (one
long part).

\section{Pre-Rendering tracks}
\subsection{The feature}
All tracks shall be able to be "pre-renderable". Pre-rendering shall
be "layered". Pre-rendering shall act like a transparent audio cache:
Audio data is (redundantly) stored, wasting memory in order to save CPU.

That is: Each track owns one or more wave-recordings of the length of
the song. If the user calls "pre-render" on a track, then this track
is played quasi-solo (see below), and the raw audio data is recorded
and stored in the "layer 0" wave recording. If the user has any effects
set up to be applied, then each effect is applied on a different layer
(creating layer 1, layer 2 etc).

This means, that also MIDI and drum tracks can have effects (which
usually only operate on audio, but we HAVE audio data because of this
prerendering).

Furthermore, MusE by default does not send MIDI events to the synthesizers
but instead just plays back the last layer of the prerecording (for
MIDI tracks), or does not pipe the audio data through the whole plugin
chain (causing cpu usage), but instead just plays back the last layer.
The hearable result shall be the same.

Once the user changes any parameter (automation data or plugins for
wave tracks, MIDI events or effect plugin stuff for MIDI tracks),
then MusE shall generate the sound for this particular track in the
"old" way (send MIDI data to synthes, or pipe audio data through plugins).
(So that the user will not even notice that MusE actually pre-renderered
stuff.) Either MusE automatically records this while playback (if possible)
or prompts the user to accordingly set up his cabling and then record
it. Or (temporarily) disables prerecording for this track, falling back
to the plain old way of generating sound.

\emph{Quasi-solo} means: For wave tracks, just solo the track. For MIDI
tracks, mute all tracks which are not on the same synth (channel?),
and mute all \emph{note} events which are not on the quasi-soloed track.
This causes MusE to still play any controller events from different
tracks, because they might have effects on the quasi-soloed track. (You
can have notes on channel 1 on one track and controller stuff on channel
1 on another track; then you would need quasi-solo to get proper results.)

\subsection{Use cases}
\paragraph{Saving CPU}
On slow systems, this is neccessary for songs with lots of, or demanding
(or both) soft synthes / plugins. Even if the synth or plugin is so
demanding that your system is not able to produce sound in real-time,
then with this feature you'll be able to use the synth (this will make
editing pretty laggish, because for a change you need to re-render at
least a part before you can listen to it, but better than being unable
to use the synth at all!)

\paragraph{Exporting as audio project}
Using pre-rendering on all tracks, you easily can export your project
as multi-track audio file (for use with Ardour or similar DAWs).
Just take the last layer of each track, and write the raw audio data
into the file, and you're done. (Maybe we are even able to write down
the raw-raw layer0 audio data plus information about used plugins and
settings etc..?)

\paragraph{Mobile audio workstations}
You might want to work a bit on your audio projects on your notebook
while you're not at home, not having access to your hardware synthesizers.
Using this feature, you could have pre-recorded the stuff in your studio
before, and now can at least fiddle around with the non-hw-synth-dependent
parts of your song, while still having your \emph{full} song with you.

\paragraph{Applying effects on MIDI tracks}
If you have many physical audio inputs, you might already be able to
apply effect chains on MIDI tracks, by wiring the synthes' audio
outputs to your soundcard's inputs, and applying the effects on
dedicated input tracks you have to create. This requires you to have
expensive hardware, and is pretty complicated, because you need one
additional track per MIDI synth.

This feature allows you to apply effects on single MIDI tracks, and not
only on full MIDI synthes, and doesn't require you to be have that
many physical audio inputs (you need to manually replug your synthes,
however).

\subsection{Possible scenarios}
\paragraph{Setting it up}
Create a wave track, MusE will allow you to set or unset prerendering
for every plugin in the plugin rack (recording the actual track is
useless because it would be a plain copy).
Create a MIDI track, MusE will ask you on which physical audio input
your synth is connected. Setting up multiple synthes on one physical 
audio in is allowed, see below.

\paragraph{Pre-rendering stuff}
When the user presses the "pre-render" button, all tracks which have
been changed since their last pre-rendering will be re-rendered.
If you have multiple hardware synthes set up as they were connected
to one physical audio input port, MusE will prompt you to first plug
the proper cable in.

\paragraph{Making changes}
Change a note in a MIDI part, move or delete a part or change automation
parameters. MusE will temporarily disable the pre-rendered information
and instead generate the sound via sending out MIDI events, piping stuff
through effect chains or similar. If you play back the whole song, or
if you manually trigger a re-rendering of a track via the context menu,
MusE will play back the stuff, record it again and re-enable the
pre-rendered information.


\section{Extensions}
\paragraph{Automatic discovery of physical audio connections}
The user plugs all (or only some) synthes' audio outs into the available
audio inputs, then runs automatic discovery. This will send MIDI events
to each synthesizer, and look on which audio in there's activity. Then
it will assume that the synthesizer is connected to that particular
audio in. Audio ins which show activity before any MIDI events were
sent are not considered, as they're probably connected to microphones
or other noise-generating non-synthes.

\paragraph{Audio export}
As described in the Use cases, MusE can allow you to export your song
in some multitrack audio format.

\paragraph{Cheap/Faked changes}
For expensive or unavailable synthes, changing the Volume midi controller,
the Pan controller or similar "easy" controllers will not trigger a
complete re-rendering, but instead "fake" the change, by changing
the volume data directly on the recorded wave. This might require some
learning and might even get pretty complicated.

\paragraph{Intelligent re-recording}
For tiny changes, MusE shall only re-render the relevant part. If you
change some MIDI notes, then begin re-recording shortly before the
changes, and end re-recording as soon as the recorded stuff doesn't
differ to much from the stuff coming from the synth. Then properly
blend the old recording with the updated part.

\end{document}
