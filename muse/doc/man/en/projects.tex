\chapter{Projects}

  \section{Project}

      A \M\ project \index{project} is represented by a folder holding 
      all files of the project. This are mainly the recorded wave files 
      and the the project file \index{project file}.
      The project file contains all information about the project.
      It also contains all midi data if exists.

      Example of a project folder structure:

      \starttyping
      ~/MusE
            projects
                  song1             
                        song1.med
                        rec1.wav
                        rec1.wca
                        rec2.wav
                        rec2.wca
                  song2
                          .             
                          .             
      \stoptyping

      In the above example {\tt song1} is the project folder 
      \index{project folder} and
      {\tt song1.med} is the project file. 
      The {\tt *.wav} files are audio recordings and the {\tt *.wca}
      files contain precomputed data used for fast screen drawing of
      waveforms.

      The path of the standard project folder 
      \index{standard project folder}  {\tt ~/MusE/projects}
      can be configured in the ''Preferences'' menu.
      

   \section{Select a project}

      After \M\ starts, first a project must be choosen or created.

      Normally the last project will be loaded. If you do not like this
      behaviour in the ''Preferences'' menu you can configure
      a standard project \index{standard project} or tell \M\ to
      always ask for a project.

   \section{Templates}

      If you enter the name of a new project in the project selection
      menu then on OK \M\ will present you a list of templates to
      choose from. The template can be a complete project but 
      without any wave data and normally without any midi data.

      \M\ sucht Templates an zwei Orten:
            im globalen \M\ Installationspfad (Factory Presets) sowie im 
                  \M\ Verzeichnis relativ zum {\tt HOME} Verzeichnis des
                        Anwenders (User Presets)

   \section{Projekt backup}

      There is no special build in function in \M\ to backup a project.
      But as all project data is contained in one folder, standard system
      tool can be used to backup.

      Project are always complete in itself and do not contain any 
      references to outside files. One execption are soundfonts \index{soundfonts}
      as used by the fluid \index{fluid} synthesizer. 
      Its recommended to manually copy these files also to the 
      project folder.

      Projects can be moved in the folder hierarchy without problems 
      as they do not contain any absolute file paths.

   \section{Wave files and samplerate}

      \index{wave files}\index{audio projecs}\index{midi projects}
      \index{samplerate}

      \M\ differences between midi projects and audio projects.
      Audio projects contain in addition to midi data wave files.
      
      Audio project have a defined samplerate and can only be loaded
      and edited if the project samplerate is identical to the
      current samplerate.

      The current sample rate is defined by the JACK audio server and
      can not be changed within \M.
                              
      If you want to import wave files with a samplerate different from
      the current sample rate, they must be converted 
      (resampled\index{resample})
