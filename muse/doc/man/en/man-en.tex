%====================================================================
%     MusE Manual
%     this is the MusE manual
%
% (C) 2006 Copyright: Werner Schweer and Others
%====================================================================

%----------------------------------------------------------
%     Setup
%----------------------------------------------------------

%\showframe
%\showsetups
\usemodule[chart]
\setupcolors[state=start]
\setupbodyfont[Palatino]
\language[en]
\enableregime[il1]

\startuseMPgraphic{FunnyFrame}
      picture p; numeric o; path a, b; pair c;
      p := textext.rt(\MPstring{FunnyFrame});
      o := BodyFontSize;
      a := unitsquare xyscaled(OverlayWidth,OverlayHeight);
      p := p shifted (20, OverlayHeight-ypart center p);
      drawoptions (withpen pencircle scaled 1pt withcolor .625red);
      b := a superellipsed .95;
      fill b withcolor .85white; draw b;
      b := (boundingbox p) superellipsed .95;
      fill b withcolor .85white; draw b;
      draw p withcolor black;
      setbounds currentpicture to a;
    \stopuseMPgraphic

\defineoverlay[FunnyFrame][\useMPgraphic{FunnyFrame}]
\defineframedtext[FunnyText][frame=off,background=FunnyFrame]
\def\StartFrame{\startFunnyText}
\def\StopFrame{\stopFunnyText}
\def\FrameTitle#1%
 {\setMPtext{FunnyFrame}{\hbox spread 1em{\hss\strut#1\hss}}}
\setMPtext{FunnyFrame}{}


\define\M{MusE}
\define[1]\Index{{\it #1}\marginpar{#1}\index{#1}}

\defineindenting[Cmdi][text=,separator=,width=fit,distance=1em]

\define[1]\Cmd{
  \Cmdi
  \framed[
     background=color,
     width=fit,
     align=right,
     backgroundcolor=lightgray,
     framecolor=blue]
     {\tt\space #1}}

\define[2]\CCmd{
  \Cmdi
  \framed[
     background=color,
     width=broad,
     align=right,
     backgroundcolor=lightgray,
     framecolor=blue]
     {\vbox{\hbox{\tt\space #1}\hbox{\tt\space #2}}}}


%\define\startdescription{\startpacked}
\define\startdescription{}
%\define\stopdescription{\stoppacked}
\define\stopdescription{}

\define[1]\Fig{\hbox{\externalfigure[pics/#1]}}

\define[2]\Screen{
      \placefigure[here][fig:#1]{#2}{\externalfigure[pics/#1][scale=1600]}
      }
\definedescription[Option][
      location=left,
      headstyle=bold,
      width=4em,
      before={\startnarrower[left]\setupblank[0pt]},
      after={\stopnarrower\setupblank}]

\definedescription[Sc][
      location=left,
      headstyle=bold,
      width=4em,
      before={\startnarrower[left]\setupblank[0pt]},
      after={\stopnarrower\setupblank}]

\definedescription[Filetype][
      location=left,
      headstyle=bold,
      width=6em,
      before={\startnarrower[left]\setupblank[0pt]},
      after={\stopnarrower\setupblank}]

\definedescription[FileList][
      location=left,
      headstyle=bold,
      width=7em,
      before={\startnarrower[left]\setupblank[0pt]},
      after={\stopnarrower\setupblank}]

\definedescription[Input][
      location=left,
      headstyle=bold,
      width=9em,
      before={\startnarrower[left]\setupblank[0pt]},
      after={\stopnarrower\setupblank}]

\definedescription[InputN][
      location=left,
      headstyle=bold,
      width=6em,
      before={\startnarrower[left]\setupblank[0pt]},
      after={\stopnarrower\setupblank}]

\define[2]\Figure{
      \placefigure
         [#1][fig:#2]{}
         {\externalfigure[pics/#2]}
         }

\define[1]\Menu{
      \placefigure[right][fig:#1]{}
         {\externalfigure[pics/#1][scale=2000]}
         }

\definedescription[Opt][location=hanging,headstyle=bold,width=broad]

%----------------------------------------------------------
%     Body
%----------------------------------------------------------

\starttext
   \language[en]
   \mainlanguage[en]
   \startstandardmakeup[doublesided=no]
      \definebodyfont[10pt,11pt,12pt][rm][tfe=Regular at 48pt]
      \tfe\setupinterlinespace
      \hfill \color[red]{\M}\par
      \hfill \color[blue]{Manual}\par
      \vfill
      \rightaligned{\externalfigure[../titlelogo.jpg]}
      \vfill
      \hfill \color[blue]{EN}\par
      \definebodyfont[10pt,11pt,12pt][rm][tfb=Regular at 24pt]
      \tfb\setupinterlinespace
      \hfill Version 1.0pre1\par
      \stopstandardmakeup

   \startstandardmakeup[page=no]
      \vfill
      \M\ is hosted on SourceForge:\par
      \type{http://mscore.sourceforge.net/}
      \blank[line]
      This document was created using \pdfTeX\ and the macro package
      \ConTeXt\.
      \blank
      \copyright 2006 Werner Schweer and Others
      \stopstandardmakeup

   \completecontent

\chapter{Introduction}
   \section{Where from and where to go}
      There's a long history of sequencers. In the beginning there were
      only analogue synthesizers, which could only be played live. The classical
      example is presumably the Moog. After the introduction of MIDI, the first
      devices appeared where music data could be saved and replayed, in short:
      Sequencers. To program these devices, i.e. to record and process music can
      rather be compared with assembler programming of computers than with making
      music. But soon the first sequencers for computers were developed. At
      first, one could only process MIDI data, since the corresponding hardware
      capacities were missing. But those developed at high speed and thus the
      quality of hardware - and what is most important in this case: sound
      cards - improved that much, that one could even do without external tape
      drives. Now multi-track audio on-line recordings were possible and there
      were even astounding possibilities for the post-editing of large projects.
      Hard to believe, but nowadays computers can do even more.
      Software-synthesizers in combination with software sequencers
      (\M\ for example!) offer apparently unlimited possibilities of
      variation. In fact, one can compose without even touching the wheel or
      the button of a "real" synthesizer. It will need time, however, until this
      software will be affordable, as the prices are immense. That's where \M\
      enters the game!

     \section{\M\ and its competitors}
      Of course there will always be "better" programs or not! This decision is
      due to strongly subjective criteria. On the one hand, it depends on what one
      is aiming at. On the other hand, it depends on individual technical knowledge
      in the field of synthesizers/sequencers. To the time that I write this article,
      there are unfortunately no commercial sequencers for Linux, not to speak of
      whole studios like Cubase or Logic. So the Linux-user can only choose between
      a parallel installation either to Mac (if one has a PPC architecture) or to
      Windows (in case of IBM Hardware). But where do those programs actually beat
      \M? Certainly not in their price because commercial, proprietary software
      is expensive. That's where programs like \M\ become interesting because
      everybody can have it for free. Furthermore they are "open source", which is
      another important point, because thereby the users get the opportunity of
      finding bugs themselves and doing away with them.

      Naturally, all projects start small. It's the same with \M. With \M,
      Linux enters a new dimension of MIDI and Audio processing. Together with
      Jack, even multi-track on-line recordings can be realized. In the the next
      few years \M\ will certainly evolve strongly and with some luck there will
      be an easier installation and more comfortable manuals.

     \section{Features of \M}

      Main features of \M:

      \startitemize[1,packed]
          \head Arranger
          	The arranger window structures a song into tracks and
            parts.

	    \head Pianoroll Editor

	    \head Drum Editor

          \head Mastertrack
            Allows you to edit tempo changes and time signatures

          \head Controller Editor
		Allows to graphically edit controller values.

          \head Recording
		Muse allows simultaneous midi and audio recording. 
            Midi can be recorded non realtime step by step.

	    \head Editing
          	Midi and audio editing can be done even during play.

	    \head LADSPA, JACK etc. compatible

        \stopitemize


\chapter{Invoking \M}
   \section{Invoking \M}
      \M\ is invoked from the command line by typing in:

      \Cmd{muse}

      Optionally, you can use command line options of the form:

      \Cmd{muse <options> <midifile>}

      \definedescription[Mops][location=left,headstyle=bold]

      \Mops{\tt <options>:}
            see below for a list of available options.
      \Mops{\tt <midifile>:}
            can be either a standard MIDI file or a \M\ file
            ({\tt *.med, *.med.gz or *.med.bz2}).

   \section{Command Line Options}

      \M\ accepts some options as listed below:

      \blank[medium]
      \Option{ -v}print version\par
      \Option{ -m}MIDI only mode\par
      \Option{ -d}debug mode: no threads, no RT\par
      \Option{ -D}debug mode: enable some debug messages\par
      \Option{ -i}debug mode: trace midi Input\par
      \Option{ -o}debug mode: trace midi Output\par
      \Option{ -p}don't load LADSPA plugins\par
      \blank[big]

      Example:

       \Cmd{muse -i mysong.med}

       will add trace midi input and open mysong.med on startup.

   \section{File types used by \M}

         \Filetype{\textasciitilde/.MusE}
            \M\ Configuration File (``\textasciitilde�� refers to your home directory
            aka "/home/tux")\par
         \Filetype{\textasciitilde/.musePrj}
            Hidden Project File; stores list of last projects\par
         \Filetype{.med}
            \M\ song file; internal \M format\par
         \Filetype{.mid}
            midi file; can be imported or exported\par
         \Filetype{.kar}
            karaoke: midi file with additional information; some types
            can be imported\par

\startappendices
   \component ../gpl.tex
   \completeindex
   \stopappendices
\stoptext
