\section{Midi Recording}
      Auf der ersten Tour werden wir mit \M\ eine Midi Spur
      mit einem externen Synthesizer, einem Yamaha S90 aufnehmen.

   \subsection{Geräte verbinden}
      Für den Anschluß von Midigeräten gibt es verschiedene Möglichkeiten.

      \subsubsection{Midi}
      Der serielle Midi-Anschluß eines Keyboards eignet sich nicht zum
      direkten Anschluß an den PC Serienport. Bei Midi handelt es sich
      um eine galvanisch per Optokoppler getrennte Stromschnittstelle,
      während der PC eine RS232 Schnittstelle mit 12 Volt Pegeln bedient.
      Ein einfaches Umrüsten der PC-Schnittstelle mit einem Optokoppler
      funktioniert nicht, da der PC die MIDI-Taktrate nicht erzeugen kann.

      Als Lösung muß ein spezielles Midi-Interface her. Meist befindet sich
      eine Midi-Schnittstelle auf der Soundkarte. Doch Vorsicht, nicht
      alle Midi-Schnittstellen sind gleich gut, z.B. sind einige alte
      Soundblaster Modelle duch unglückliches Hardware-Design nicht in der
      Lage, die volle Midi Transferrate beim Senden von Midi Daten zu
      erreichen.

      \subsubsection{USB}
      Viele aktuelle Midigeräte besitzen einen USB Anschluß. USB ist schnell
      und flexibel und damit die Schnittstelle der Wahl.

      \subsubsection{ToHost}
      Einige ältere Midigeräte bieten einen einfachen PC Anschluß über 
      einen normalen Serienport (''ToHost'' Schnittstelle genannt). Es
      handelt sich um einen normalen RS232 Anschluß, der die PC-üblichen
      Übertragungsraten versteht.

   \subsection{Neues Projekt erzeugen}
      Dann kanns losgehen. Wir starten \M\ und es erscheint das
      zuletzt bearbeitete Projekt so wie wir es verlassen haben.

      Um ein neues Projekt anzulegen clicken wir auf das Projekt
      Icon und der Projektdialog erscheint:

      \Fig{select_project}

      Wenn \M\ zum allererstenmal gestartet wird, erscheint der
      Projektdialog natürlich sofort, da es ja kein letztes Projekt
      gibt. Zum Projektdialog gelangen wir auch über den Projekmenü
      Eintrag ''Öffnen'' oder wer es ganz eilig hat tippt einfach
      {\tt Ctrl+O}.

      In der ''Projekt'' Eingabezeile geben wir nun einen Projektnamen
      für unser erstes Projekt ein und bestätigen dann durch clicken
      des ''Ok'' Buttons.

      \M\ fragt nun nach einem Template, mit dem das Projekt
      initialisiert werden soll:

      \Fig{select_template}

      Wir selektieren kein Template und clicken einfach ''Ok''.
      \M\ zeigt dann ein leeres Projekt:

      \Fig{main0}

   \subsection{Setup}

      \subsubsection{Spuren anlegen}

      Zunächst erzeugen wir einen ''MidiInput'' Track und checken,
      ob das Keyboard richtig angeschlossen ist.

      \subsubsection{Routing}

      \subsubsection{Midi Instrument}
      Im ''MidiOutPort'' Strip kann ein Midi Instrument ausgewählt werden.
      Dies ermöglicht es \M\ u.A. die Instrument-Patches anstatt per Nummer 
      per Name anzuzeigen. Instrumente werden in einer ''{\tt *.idf}''
      (Instrument Definition File) beschrieben. Dies sind normale XML Text
      Dateien, die für jedes angeschlossene Gerät individuell erstellt werden
      können. In einer {\tt *.idf} Datei kann in einem ''Init'' Eintrag eine
      Serien von Midi Events definiert werden, die immer dann gesendet werden,
      wenn in \M\ der {\tt Midi->InitInstr} Button aktiviert wird. Wir nutzen diese
      Möglichkeit, um den S90 zu initialisieren.

      \subsubsection{Local Off}
      Wird ein Synthesizer ''standalone'' betrieben, dann ist die Tastatur
      intern direkt mit dem Tongenerator verbunden. Am Computer angeschlossen
      möchten wir jedoch Tastatur und Tongenerator als unabhängige Teile
      benutzen. Note On/Off Events sollen von der Tastatur nur zum Computer
      geschickt werden und nicht auch gleich zum internen Tongenerator.
      Der Tongenerator soll wiederum nur auf den Midi Input Anschluß hören
      und nicht auf die Tastatur. Für den S90 kann dieser ''Local Off''
      Mode durch Senden des SysEx Strings

      \starttyping
            0xf0 0x10 0x6b 0x00 0x00 0x09 0x00
      \stoptyping

      eingeschaltet werden.

      \subsubsection{Sequencer Mode}
      Der S90 kenn Yamaha-Typisch verschiedene Modi. Wir benutzen den
      Sequencer Mode, in dem jedem Midi Kanal ein eigenes Instrument
      zugeordnet werden kann. Dieser Modus wird mit dem SysEx String

      \starttyping
            0xf0 0x43 0x10 0x6b 0x0a 0x00 0x00 0x00
      \stoptyping

      eingeschaltet. Beide SysEx Kommandos sind unter dem ''init'' Label in
      der {\tt Yamaha-S90.idf} Datei eingetragen.

      \subsubsection{Midi Input Filter}
      Die Tastatur des Yamaha S90 kann ''After Touch'' Events erzeugen, d.h.
      nach dem Drücken einer Taste werden fortlaufend Events über den
      Anpreßdruck der Taste erzeugt. Diese Informationen benötigen wir
      nicht und filtern deshalb alle After Touch Events gleich bei der
      Aufnahme mittels des Midi Input Filters aus.

      \Fig{midifilter}

      \subsection{Aufnehmen}
