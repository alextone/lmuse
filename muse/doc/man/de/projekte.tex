\chapter{Projekte}

  \section{Projekte}

      Ein \M\ Projekt \index{Projekt} ist ein Ordner, der alle 
      Dateien des Projekts enthält. Dies sind im wesentlichen die 
      aufgenommenen Wavedateien und die Projektdatei \index{Projektdatei}.
      Die Projektdatei enthält alle
      Informationen über das Projekt sowie alle Midi Daten soweit
      vorhanden.

      Beispiel einer Projekt Ordnerstruktur:

      \starttyping
      ~/MusE
            projects
                  song1             
                        song1.med
                        rec1.wav
                        rec1.wca
                        rec2.wav
                        rec2.wca
                  song2
                          .             
                          .             
      \stoptyping

      Im oberen Beispiel ist {\tt song1} der Projektordner
      \index{Projektordner} und
      {\tt song1.med} die Projektdatei. Die {\tt *.wav} Dateien sind
      Audio Aufzeichnungen und die zugehörigen {\tt *.wca} Dateien 
      enthalten vorberechnete Daten zum schnellen Zeichnen der 
      Wellenformdarstellung auf den Bildschirm.

      Der Standard-Projektordner \index{Standard-Projektordners}
      {\tt ~/MusE/projects} kann im ''Präferenzen'' Menü eingestellt 
      werden.

   \section{Projekt auswählen}

      Nach dem Start von \M\ muß immer zunächst ein Projekt geladen oder
      ein neues Projekt erzeugt werden.

      \Screen{select_project}{Projekt auswählen}

      Normalerweise wird nach dem Start von \M\ das zuletzt bearbeitete
      Projekt geladen. Wer das nicht mag kann unter ''Präferenzen'' 
      einstellen, ob beim Start ein Standardprojekt \index{Standardprojekt}
      geladen oder immer zunächst nach einen Projekt gefragt werden soll.

   \section{Templates}

      Wird bei der Projektauswahl ein neues Projekt ausgewählt, dann
      wird im nächsten Schritt ein Dialog gezeigt, in dem ein Template
      aus einer Liste von verfügbaren Templates ausgewählt werden kann.
      Ein Template bestimmt die Grundeinstellungen für ein neues
      Projekt. Ein Template kann ein komplettes Projekt sein, welches
      jedoch keine Audiodaten enthalten kann und welches in der Regel
      auch keine Midi Daten enthält.

      \Screen{select_template}{Template auswählen}

      \M\ sucht Templates an zwei Orten:
      im globalen \M\ Installationspfad (Factory Presets) sowie im 
      \M\ Verzeichnis relativ zum {\tt HOME} Verzeichnis des
      Anwenders (User Presets)

   \section{Projekt sichern (Backup)}

      Es gibt keine in \M\ eingebaute Funktion zum Sichern von Projekten.
      Da sich aber alle Projektdaten in einem Ordner befinden,
      können Standard-Tools zum Sichern verwendet werden.

      Projekte sind immer in sich komplett und enthalten keine Verweise
      auf Dateien außerhalb des Projektordners. Dies gilt nicht
      für Soundfonts \index{Soundfonts}, die z.B. vom Fluid-Synthesizer 
      Plugin verwendet werden. 
      Es wird empfohlen, diese Soundfonts\index{Soundfonts} 
      auch als Kopie im Projektordner abzulegen.

      Projekte können auf der Platte ohne Probleme verschoben werden,
      da sie keine absoluten Pfade enthalten.

   \section{Wavedateien und Samplerate}

      \index{Wavedateien}\index{Audioprojekte}\index{Midiprojekte}
      \index{Samplerate}
      MusE unterscheidet zwischen Midiprojekten und Audioprojekten.
      Audioprojekte enthalten zusätzlich zu eventuell vorhandenen
      Mididaten Wavedateien.
      
      Audioprojekte sind immer mit einer bestimmten Samplerate
      verbunden und können nur geladen und bearbeitet werden, wenn
      die Projekt Samplerate mit der aktuellen Samplerate übereinstimmt.

      Die aktuelle Samplerate wird immer vom JACK Server bestimmt und
      kann von \M\ nicht verändert werden.

      Werden Wavedateien mit einer anderen Samplerate als der aktuellen
      importiert, so müssen sie konvertiert (resampled\index{resample}) 
      werden.

