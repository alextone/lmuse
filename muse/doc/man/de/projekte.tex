\chapter{Projekte}

  \section{Was ist ein Projekt?}

      Ein \M\ Projekt \index{Projekt} ist ein Unterverzeichnis, 
      welches alle 
      Dateien des Projekts enth�lt. Dies sind im wesentlichen die 
      aufgenommenen Wavedateien und die Projektdatei \index{Projektdatei}.
      Die Projektdatei enth�lt alle
      Informationen �ber das Projekt sowie alle Midi Daten soweit
      vorhanden.

      Beispiel einer Projekt Verzeichnisstruktur:

      \starttyping
      ~/muse
            songs
                  song1             
                        song1.med
                        rec1.wav
                        rec1.wca
                        rec2.wav
                        rec2.wca
                  song2
                          .             
                          .             
      \stoptyping

      Im oberen Beispiel ist {\tt song1} das Projektverzeichnis 
      \index{Projektverzeichnis} und
      {\tt song1.med} die Projektdatei. Die {\tt *.wav} Dateien sind
      Audio Aufzeichnungen und die zugeh�rigen {\tt *.wca} Dateien 
      enthalten vorberechnete Daten zum schnellen Zeichnen der 
      Wellenformdarstellung auf den Bildschirm.

      Das Standard-Projektverzeichnis \index{Standard-Projektverzeichnis}
      {\tt ~/muse/songs} kann im
      ''Pr�ferenzen'' Men� eingestellt werden.

   \section{Projekt ausw�hlen}

      Nach dem Start von \M\ mu� immer zun�chst ein Projekt geladen oder
      ein neues Projekt erzeugt werden.

      Normalerweise wird nach dem Start von \M\ das zuletzt bearbeitete
      Projekt geladen. Wer das nicht mag kann unter ''Pr�ferenzen'' 
      einstellen, ob beim Start ein Standardprojekt \index{Standardprojekt}
      geladen oder immer zun�chst nach einen Projekt gefragt werden soll.

          
   \section{Projekt sichern (Backup)}

      Es gibt keine in \M\ eingebaute Funktion zum Sichern von Projekten.
      Da sich aber alle Projektdaten in einem Unterverzeichnis befinden,
      k�nnen Standard-Tools zum sichern verwendet werden.

      Projekte sind immer in sich komplett und enthalten keine Verweise
      auf Dateien au�erhalb des Projektverzeichnisses. Dies gilt nicht
      f�r Soundfonts \index{Soundfonts}, die z.B. vom Fluid-Synthesizer 
      Plugin verwendet
      werden. Es wird empfohlen, diese Soundfonts auch als Kopie im
      Projektverzeichnis abzulegen.

      Projekte k�nnen auf der Platte ohne Probleme verschoben werden,
      da sie keine absoluten Pfade enthalten.

   \section{Wavedateien und Samplerate}

      \index{Wavedateien}\index{Audioprojekte}\index{Midiprojekte}
      \index{Samplerate}
      MusE unterscheidet zwischen Midiprojekten und Audioprojekten.
      Audioprojekte enthalten zus�tzlich zu eventuell vorhandenen
      Mididaten Wavedateien.
      
      Audioprojekte sind immer mit einer bestimmten Samplerate
      verbunden und k�nnen nur geladen und bearbeitet werden, wenn
      die Projekt Samplerate mit der aktuellen Samplerate �bereinstimmt.

      Die aktuelle Samplerate wird immer vom JACK Server bestimmt und
      kann von \M\ nicht ver�ndert werden.

      Werden Wavedateien mit einer anderen Samplerate als der aktuellen
      importiert, so m�ssen sie konvertiert (resampled\index{resample}) 
      werden.

