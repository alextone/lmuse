%====================================================================
%     MusE Manual
%     dies ist das MusE Manual
%
% (C) 2006 Copyright: Werner Schweer und Andere
%====================================================================

%----------------------------------------------------------
%     Setup
%----------------------------------------------------------

%\showframe
%\showsetups
\usemodule[chart]
\setupcolors[state=start]
\setupbodyfont[Palatino]
\language[de]

% typeset in unicode (utf8)
% \enableregime[il1]
\enableregime[utf]

\startuseMPgraphic{FunnyFrame}
      picture p; numeric o; path a, b; pair c;
      p := textext.rt(\MPstring{FunnyFrame});
      o := BodyFontSize;
      a := unitsquare xyscaled(OverlayWidth,OverlayHeight);
      p := p shifted (20, OverlayHeight-ypart center p);
      drawoptions (withpen pencircle scaled 1pt withcolor .625red);
      b := a superellipsed .95;
      fill b withcolor .85white; draw b;
      b := (boundingbox p) superellipsed .95;
      fill b withcolor .85white; draw b;
      draw p withcolor black;
      setbounds currentpicture to a;
    \stopuseMPgraphic

\defineoverlay[FunnyFrame][\useMPgraphic{FunnyFrame}]
\defineframedtext[FunnyText][frame=off,background=FunnyFrame]
\def\StartFrame{\startFunnyText}
\def\StopFrame{\stopFunnyText}
\def\FrameTitle#1%
 {\setMPtext{FunnyFrame}{\hbox spread 1em{\hss\strut#1\hss}}}
\setMPtext{FunnyFrame}{}


\define\M{MusE}
\define[1]\Index{{\it #1}\marginpar{#1}\index{#1}}

\defineindenting[Cmdi][text=,separator=,width=fit,distance=1em]

\define[1]\Cmd{
  \Cmdi
  \framed[
     background=color,
     width=fit,
     align=right,
     backgroundcolor=lightgray,
     framecolor=blue]
     {\tt\space #1}}

\define[2]\CCmd{
  \Cmdi
  \framed[
     background=color,
     width=broad,
     align=right,
     backgroundcolor=lightgray,
     framecolor=blue]
     {\vbox{\hbox{\tt\space #1}\hbox{\tt\space #2}}}}

%\define\startdescription{\startpacked}
\define\startdescription{}
%\define\stopdescription{\stoppacked}
\define\stopdescription{}

\define[1]\Fig{\blank\hbox{\externalfigure[#1]}\blank}

\define[2]\Screen{
      \placefigure[here][fig:#1]{#2}{\externalfigure[#1][scale=1600]}
      }
\definedescription[Option][
      location=left,
      headstyle=bold,
      width=4em,
      before={\startnarrower[left]\setupblank[0pt]},
      after={\stopnarrower\setupblank}]

\definedescription[Filetype][
      location=left,
      headstyle=bold,
      width=6em,
      before={\startnarrower[left]\setupblank[0pt]},
      after={\stopnarrower\setupblank}]

\definedescription[FileList][
      location=left,
      headstyle=bold,
      width=7em,
      before={\startnarrower[left]\setupblank[0pt]},
      after={\stopnarrower\setupblank}]

\definedescription[Input][
      location=left,
      headstyle=bold,
      width=9em,
      before={\startnarrower[left]\setupblank[0pt]},
      after={\stopnarrower\setupblank}]

\definedescription[InputN][
      location=left,
      headstyle=bold,
      width=6em,
      before={\startnarrower[left]\setupblank[0pt]},
      after={\stopnarrower\setupblank}]

\define[2]\Figure{
      \placefigure
         [#1][fig:#2]{}
         {\externalfigure[pics/#2]}
         }

\define[1]\Menu{
      \placefigure[right][fig:#1]{}
         {\externalfigure[pics/#1][scale=2000]}
         }

\definedescription[Opt][location=hanging,headstyle=bold,width=broad]

\setuptyping[before=\blank\startbackground, after=\stopbackground\blank] % source code with background
\setupwhitespace[medium]

\component figurepath.tex

%----------------------------------------------------------
%     Body
%----------------------------------------------------------

\starttext
   \language[de]
   \mainlanguage[de]
   \startstandardmakeup[doublesided=no]
      \definebodyfont[10pt,11pt,12pt][rm][tfe=Regular at 48pt]
      \tfe\setupinterlinespace
      \hfill \color[red]{\M}\par
      \hfill \color[blue]{Manual}\par
      \vfill
      \rightaligned{\externalfigure[titlelogo]}
      \vfill
      \hfill \color[blue]{DE}\par
      \definebodyfont[10pt,11pt,12pt][rm][tfb=Regular at 24pt]
      \tfb\setupinterlinespace
      \hfill Version 1.0pre1\par
      \stopstandardmakeup

   \startstandardmakeup[page=no]
      \vfill
      \M\ wird von SourceForge gehostet:\par
      \type{http://sourceforge.net/projects/lmuse}
      \blank[line]
      Dieses Dokument wurde mit \pdfTeX\ und dem Macro Paket
      \ConTeXt\ erstellt.
      \blank
      \copyright 2006 Werner Schweer und Andere\par
      Titelgrafik \copyright Joachim Schiele 
      \stopstandardmakeup

   \completecontent

\chapter{\M\ Quickstart}
   \component midirecording.tex
   \section{Audio Playback}
   \section{Audio Recording}

\component projekte.tex
\component struktur.tex
\component miditracks.tex

\chapter{Automation}
      \component automation.tex

\chapter{Funktionsreferenz}
   \section{Hauptfenster}
      \subsection{Menüs}
            Projekt
                  Open Project
                  Open Recent
                  Save Project
                  Import Midifile
                  Export Midifile
                  Import Wavefile
                  Quit
            Edit
                  Undo
                  Redo
                  Cut
                  Copy
                  Paste
                  Delete Parts
                  Delete Selected Tracks
                  Add Track
                        Add Midi Track
                        Add Midi Output
                        Add Midi Input
                        Add Midi Generator
                        Add Wave Track
                        Add Audio Output
                        Add Audio Group
                        Add Audio Input
                        Add Soft Synth
                  Select
                        Select All
                        Deselect All
                        Invert Selection
                        Inside Loop
                        Outside Loop
                        All Parts on Track
                  Pianoroll
                  Midi Tracker
                  Drums
                  List
                  Master Track
                  Project Properties

            View
                  Transport Panel
                  Bigtime Window
                  Mixer 1
                  Mixer 2
                  Marker

            Structure
                  Global Cut
                  Global Insert
                  Global Split
                  Copy Range
                  Cut Events
            Midi
                  Edit Instrument
                  Reset Instr.
                  Init Instr.
                  local off
            Audio
                  Bounce To Track
                  Bounce To File
                  Restart Audio
            Settings
                  Configure Shortcuts
                  Follow Song
                        dont follow Song
                        follow page
                        follow continuous
                  Midi Sync
                  Midi File Export
                  Preferences

            Help
                  Manual
                  MusE Homepage
                  Report Bug
                  About MusE
                  About Qt
                  What's This

      \subsection{Werkzeugleisten}
            Project Buttons
                  Open
                  Save
                  Whats This
                  Undo
                  Redo
            Edit Tools
                  Zeiger
                  Stift
                  Radiergummi
                  Schere
                  Kleber
                  Linientool
                  Mute
            Transport
                  Loop
                  Punch In
                  Punch Out
                  Gehe nach Anfang
                  Zurückspulen
                  Vorspulen
                  Stop
                  Play
                  Record
            Panic
            Arranger
                  Cursor
                  Snap
                  Len
                  Pitch
                  Tempo
                  50%
                  N
                  200%
            Trackliste
                  Einstellungen
                  m     global Mute
                  s     global Solo
                  aR    global AutomationRead
                  aW    global AutomationWrite

            Info Button
            Mixer Button

      \subsection{Trackliste}
            Midi Input
            Midi Output
            Midi Track
            Midi Synthesizer
            Audio Track
            Audio Input
            Audio Output
            Audio Group
            Audio Synthesizer

      \subsection{Arranger}
            Lineal
            Marker
            Locator

      
   \section{Globale Editor Funktionen}
      \subsection{Globaler Schnitt}
         Die ''Globaler Schnitt''-Funktion enfernt auf allen Spuren den Bereich 
         zwischen der linken und rechten Marke. Nachfolgende Events rücken 
         dabei auf. Auch Markerposition und Tempoevents werden
         verschoben. Lediglich stumm geschaltete Spuren werden nicht 
         berücksichtigt.

      \subsection{Global Einfügen}
         Mit der ''Global Einfügen''-Funktion kann ein leerer Bereich eingefügt 
         werden. Der Bereich wird an der Position des linken Markers 
         eingefügt. Die Dauer des eingefügten Bereichs entspricht dem Abstand 
         zwischen linker und rechter Marke. Marker und Tempoevents hinter der 
         Einfügeposition werden entsprechend verschoben.
         Stumm geschaltete Spuren werden nicht verändert.

      \subsection{Global Splitten}
         Die ''Global Splitten''-Funktion entspricht der Scheren-Funktion für 
         einzelne Parts. Im Gegensatz zur Scheren-Funktion wird jedoch nicht 
         ein Part, sondern alle Parts an der aktuellen Markerposition 
         geschnitten. Dies gilt nicht für stummgeschaltete Spuren.

      \subsection{Bereich Kopieren}                 
         Mit der ''Global Kopieren''-Funktion können komplette Teile eines 
         Songs an eine andere Stelle kopiert werden:

         \startitemize[packed]
         \item stellen Sie die linke Marke an den Anfang des zu kopierenden 
               Bereichs
         \item stellen Sie die rechte Marke an das Ende des zu kopierenden 
               Bereichs
         \item setzen Sie den Lokator an die Einfgestelle
         \item wählen Sie ''Copy Range'' aus dem ''Struktur'' Menü.
         \stopitemize

         Wie bei allen anderen Strukturkommandos werden stummgeschaltete 
         Spuren nicht berücksichtigt.

   \section{Arranger}
      \subsection{Spurtypen}
         \startitemize[packed]
            \item Midispur
            \item Audiospur
            \item Midiinput
            \item Midioutput
            \item Midigenerator
            \item Audioinput
            \item Audiooutput
            \item Audiogruppe
            \item Software Synthesizer
         \stopitemize

      \subsection{Spur Operationen}
         \startitemize[packed]
            \item Spur erzeugen
            \item Spur löschen
            \item Spur verschieben
            \item Spur kopieren
         \stopitemize

      \subsection{Part Operationen}
         \startitemize[packed]
            \item Part erzeugen
            \item Part löschen
            \item Part verschieben
            \item Part kopieren
            \item Part clonen
            \item Part declonen
            \item Part nach Vorne verlängern
            \item Part nach Hinten verlängern
            \item Part stummschalten
            \item Part einfärben
         \stopitemize
      
   \section{Pianoroll Editor}
   \section{Drum Editor}
   \section{Master Editor}

\chapter{Installation}
   \component installation.tex

\startappendices
   \component ../gpl.tex
   \completeindex
   \stopappendices
\stoptext
        
