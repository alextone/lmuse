%====================================================================
%     MusE Manual
%     dies ist das MusE Manual
%
% (C) 2006 Copyright: Werner Schweer und Andere
%====================================================================

%----------------------------------------------------------
%     Setup
%----------------------------------------------------------

%\showframe
%\showsetups
\usemodule[chart]
\setupcolors[state=start]
\setupbodyfont[Palatino]
\language[de]
\enableregime[il1]

\startuseMPgraphic{FunnyFrame}
      picture p; numeric o; path a, b; pair c;
      p := textext.rt(\MPstring{FunnyFrame});
      o := BodyFontSize;
      a := unitsquare xyscaled(OverlayWidth,OverlayHeight);
      p := p shifted (20, OverlayHeight-ypart center p);
      drawoptions (withpen pencircle scaled 1pt withcolor .625red);
      b := a superellipsed .95;
      fill b withcolor .85white; draw b;
      b := (boundingbox p) superellipsed .95;
      fill b withcolor .85white; draw b;
      draw p withcolor black;
      setbounds currentpicture to a;
    \stopuseMPgraphic

\defineoverlay[FunnyFrame][\useMPgraphic{FunnyFrame}]
\defineframedtext[FunnyText][frame=off,background=FunnyFrame]
\def\StartFrame{\startFunnyText}
\def\StopFrame{\stopFunnyText}
\def\FrameTitle#1%
 {\setMPtext{FunnyFrame}{\hbox spread 1em{\hss\strut#1\hss}}}
\setMPtext{FunnyFrame}{}


\define\M{MusE}
\define[1]\Index{{\it #1}\marginpar{#1}\index{#1}}

\defineindenting[Cmdi][text=,separator=,width=fit,distance=1em]

\define[1]\Cmd{
  \Cmdi
  \framed[
     background=color,
     width=fit,
     align=right,
     backgroundcolor=lightgray,
     framecolor=blue]
     {\tt\space #1}}

\define[2]\CCmd{
  \Cmdi
  \framed[
     background=color,
     width=broad,
     align=right,
     backgroundcolor=lightgray,
     framecolor=blue]
     {\vbox{\hbox{\tt\space #1}\hbox{\tt\space #2}}}}


%\define\startdescription{\startpacked}
\define\startdescription{}
%\define\stopdescription{\stoppacked}
\define\stopdescription{}

\define[1]\Fig{\hbox{\externalfigure[pics/#1]}}

\define[2]\Screen{
      \placefigure[here][fig:#1]{#2}{\externalfigure[pics/#1][scale=1600]}
      }
\definedescription[Option][
      location=left,
      headstyle=bold,
      width=4em,
      before={\startnarrower[left]\setupblank[0pt]},
      after={\stopnarrower\setupblank}]

\definedescription[Filetype][
      location=left,
      headstyle=bold,
      width=6em,
      before={\startnarrower[left]\setupblank[0pt]},
      after={\stopnarrower\setupblank}]

\definedescription[FileList][
      location=left,
      headstyle=bold,
      width=7em,
      before={\startnarrower[left]\setupblank[0pt]},
      after={\stopnarrower\setupblank}]

\definedescription[Input][
      location=left,
      headstyle=bold,
      width=9em,
      before={\startnarrower[left]\setupblank[0pt]},
      after={\stopnarrower\setupblank}]

\definedescription[InputN][
      location=left,
      headstyle=bold,
      width=6em,
      before={\startnarrower[left]\setupblank[0pt]},
      after={\stopnarrower\setupblank}]

\define[2]\Figure{
      \placefigure
         [#1][fig:#2]{}
         {\externalfigure[pics/#2]}
         }

\define[1]\Menu{
      \placefigure[right][fig:#1]{}
         {\externalfigure[pics/#1][scale=2000]}
         }

\definedescription[Opt][location=hanging,headstyle=bold,width=broad]

\component figurepath.tex

%----------------------------------------------------------
%     Body
%----------------------------------------------------------

\starttext
   \language[de]
   \mainlanguage[de]
   \startstandardmakeup[doublesided=no]
      \definebodyfont[10pt,11pt,12pt][rm][tfe=Regular at 48pt]
      \tfe\setupinterlinespace
      \hfill \color[red]{\M}\par
      \hfill \color[blue]{Manual}\par
      \vfill
      \rightaligned{\externalfigure[titlelogo]}
      \vfill
      \hfill \color[blue]{DE}\par
      \definebodyfont[10pt,11pt,12pt][rm][tfb=Regular at 24pt]
      \tfb\setupinterlinespace
      \hfill Version 1.0pre1\par
      \stopstandardmakeup

   \startstandardmakeup[page=no]
      \vfill
      \M\ wird von SourceForge gehostet:\par
      \type{http://sourceforge.net/projects/lmuse}
      \blank[line]
      Dieses Dokument wurde mit \pdfTeX\ und dem Macro Paket
      \ConTeXt\ erstellt.
      \blank
      \copyright 2006 Werner Schweer und Andere
      \stopstandardmakeup

   \completecontent

\chapter{Einf�hrung}
   \section{Was ist \M?}
      \M\ ist ein ein ''open source'' Programm, ver�ffentlicht unter der GNU General
      Public License (GPL) (siehe Anhang).

\component miditracks.tex
\component projekte.tex

\startappendices
   \component ../gpl.tex
   \completeindex
   \stopappendices
\stoptext
